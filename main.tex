\documentclass{article}

\usepackage[spanish]{babel}
\usepackage[a4paper,top=2.54cm,bottom=2.54cm,left=2.54cm,right=2.54cm]{geometry}

\usepackage{mathptmx}

\usepackage{amsmath}
\usepackage{graphicx}
\usepackage{setspace}
\usepackage[colorlinks=true, allcolors=blue]{hyperref}

\usepackage[style=apa, backend=biber]{biblatex}
\DeclareLanguageMapping{spanish}{spanish-apa}
\addbibresource{sample.bib}

\doublespacing

\title{\textbf{BUENAS PRÁCTICAS EN EL DESARROLLO DE APLICACIONES REACTIVAS}}
\author{%
    \begin{tabular}{c}
        \textit{E.A. Girón Leonardo} \\
        \textit{7690-21-218 Universidad Mariano Gálvez} \\
        \textit{Seminario de Tecnologías} \\
        \href{mailto:egironl12@miumg.edu.gt}{\uline{egironl12@miumg.edu.gt}}
    \end{tabular}%
}
\date{}

\begin{document}
\maketitle

\begin{abstract}
El desarrollo de software moderno exige arquitecturas capaces de responder a una gran carga de usuarios, ofrecer tiempos de respuesta mínimos y adaptarse a entornos dinámicos. En este contexto surgen las aplicaciones reactivas, diseñadas para responder de forma eficiente a eventos, errores y cambios de estado. Este documento explora los fundamentos de las aplicaciones reactivas, sus principios, tecnologías comunes y, sobre todo, las buenas prácticas recomendadas para su desarrollo. Se incluyen referencias recientes y casos de uso que validan la importancia de adoptar este enfoque en entornos reales.
\end{abstract}

\textbf{Palabras clave:} aplicaciones reactivas, programación reactiva, buenas prácticas, arquitectura escalable, asincronía.

\section{Introducción}

El auge de las aplicaciones modernas, especialmente aquellas que se ejecutan en tiempo real o que atienden a miles de usuarios simultáneamente, ha hecho evidente la necesidad de adoptar nuevos paradigmas arquitectónicos. La programación reactiva ha cobrado fuerza como una solución eficiente para enfrentar retos de concurrencia, escalabilidad, y tolerancia a fallos.

A diferencia del enfoque tradicional sincrónico, las aplicaciones reactivas se basan en la gestión eficiente de flujos de datos asincrónicos y eventos. Esta capacidad les permite ofrecer un rendimiento notable incluso bajo condiciones de alta demanda, haciendo uso óptimo de los recursos del sistema.

\section{Marco Teórico}

\subsection{Definición de aplicaciones reactivas}

Una aplicación reactiva está diseñada para reaccionar ante eventos, datos y situaciones inesperadas, como errores o picos de carga, sin comprometer su funcionalidad ni su rendimiento. Este tipo de aplicaciones opera bajo una lógica de programación orientada a eventos, donde cada componente responde de forma autónoma y no bloqueante.

Estas aplicaciones permiten construir soluciones que mantienen su estabilidad ante interrupciones o cambios del entorno, ya que su diseño facilita el aislamiento de fallos y la recuperación parcial del sistema sin afectar la experiencia del usuario. Según \textcite{toro2021}, las aplicaciones reactivas permiten construir sistemas no bloqueantes y escalables.Como se menciona en \parencite{villanueva2020}, el enfoque reactivo está orientado a eventos y mejora el rendimiento general del software.

\subsection{El manifiesto reactivo}

El Reactive Manifesto (2014) establece los principios fundamentales que deben regir una aplicación reactiva:

\begin{itemize}
    \item \textbf{Responsiva}: el sistema debe responder en tiempos predecibles.
    \item \textbf{Resiliente}: debe mantenerse operativo ante fallos internos o externos.
    \item \textbf{Elástica}: debe adaptarse dinámicamente a cambios en la carga.
    \item \textbf{Orientada a mensajes}: la comunicación entre componentes debe ser asincrónica y desacoplada.
\end{itemize}

Estos principios permiten construir sistemas preparados para enfrentar incertidumbres y demandas variables, como ocurre en aplicaciones de comercio electrónico, servicios financieros o plataformas de streaming.

\subsection{Tecnologías asociadas}

El desarrollo de aplicaciones reactivas puede implementarse en diversos lenguajes y frameworks. Algunas de las tecnologías más populares incluyen:

\begin{itemize}
    \item \textbf{Spring WebFlux}: Framework en Java basado en el servidor no bloqueante Netty.
    \item \textbf{ReactJS y Redux}: Usados en la construcción de interfaces de usuario reactivas, con control de estado centralizado.
    \item \textbf{RxJS (Reactive Extensions)}: Biblioteca para JavaScript que permite el manejo de flujos de datos mediante observables.
    \item \textbf{Flutter con Bloc o Riverpod}: Enfoques utilizados en aplicaciones móviles que permiten una gestión reactiva del estado.
    \item \textbf{Akka y Vert.x}: Plataformas orientadas a la construcción de sistemas distribuidos, con procesamiento basado en actores y mensajes.
\end{itemize}

\section{Buenas prácticas en el desarrollo de aplicaciones reactivas}

\subsection{Estructura modular y componentes desacoplados}

Una buena práctica esencial consiste en diseñar la aplicación en módulos pequeños, independientes y reutilizables. El desacoplamiento entre componentes permite que estos se comuniquen mediante mensajes asincrónicos, lo que mejora la escalabilidad y la mantenibilidad del sistema.

\subsection{Control centralizado del estado}

El manejo del estado debe ser predecible y coherente. Bibliotecas como Redux o Context API en React, y Bloc o Riverpod en Flutter, permiten mantener una única fuente de verdad en la aplicación. Esto facilita depurar errores, aplicar pruebas unitarias y mejorar la sincronización entre vistas y lógica de negocio.

\subsection{Evitar suscripciones no gestionadas}

Las suscripciones a flujos deben ser controladas cuidadosamente. Dejar suscripciones activas de forma indefinida puede causar pérdidas de memoria y fugas de recursos. Por ello, se recomienda usar operadores como \texttt{takeUntil}, \texttt{first}, o cerrar manualmente los \texttt{subscriptions} cuando ya no se necesitan.

\subsection{Manejo adecuado de asincronía}

Los sistemas reactivos operan sobre eventos asincrónicos. Por ello, es fundamental evitar operaciones bloqueantes (como \texttt{sleep} o lecturas sincrónicas) y emplear flujos no bloqueantes. Operadores como \texttt{debounce}, \texttt{mergeMap} y \texttt{switchMap} permiten gestionar correctamente las respuestas del sistema ante múltiples eventos simultáneos.

\subsection{Resiliencia y tolerancia a fallos}

Para garantizar que la aplicación se mantenga operativa ante errores, se debe implementar un manejo de errores adecuado en todos los flujos. Se recomienda el uso de patrones como \textit{retry}, \textit{fallback}, y \textit{circuit breaker} para que el sistema se recupere sin intervención del usuario.

\subsection{Monitoreo y análisis continuo}

Una buena práctica es instrumentar la aplicación con herramientas de monitoreo que permitan observar métricas clave como tiempos de respuesta, errores, tasa de eventos y uso de recursos. Esto permite anticipar fallos, optimizar cuellos de botella y mejorar el rendimiento.

\section{Aplicaciones prácticas y escenarios de uso}

Las aplicaciones reactivas son ideales en escenarios donde la concurrencia y la alta disponibilidad son fundamentales. Algunos ejemplos incluyen:

\begin{itemize}
    \item \textbf{Sistemas de mensajería y chat en tiempo real}: donde múltiples usuarios se comunican simultáneamente.
    \item \textbf{Plataformas de video en streaming}: que deben gestionar eventos de reproducción, buffering y cambio de resolución en tiempo real.
    \item \textbf{APIs de servicios financieros}: como pasarelas de pago o sistemas bancarios que procesan millones de transacciones al día.
    \item \textbf{Aplicaciones IoT}: que manejan grandes volúmenes de datos provenientes de sensores distribuidos geográficamente.
\end{itemize}

\section{Conclusiones}

Las aplicaciones reactivas representan un avance significativo en la forma de diseñar y construir sistemas modernos, capaces de responder eficientemente a demandas complejas. Adoptar buenas prácticas como el desacoplamiento de componentes, la gestión adecuada del estado, el uso de patrones de resiliencia y el monitoreo continuo permite construir soluciones más robustas, escalables y sostenibles a largo plazo.

El enfoque reactivo no es una moda, sino una evolución lógica del desarrollo de software frente a las necesidades actuales del mercado. A medida que la infraestructura digital crece, este paradigma se convierte en una herramienta esencial para garantizar experiencias de usuario fluidas, estables y seguras.

\printbibliography

\end{document}
